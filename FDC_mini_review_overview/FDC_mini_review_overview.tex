\documentclass{beamer}

\usetheme{Boadilla}

\newcommand{\bi}{\begin{itemize}}
\newcommand{\ei}{\end{itemize}}
\newcommand{\be}{\begin{enumerate}}
\newcommand{\ee}{\end{enumerate}}
\newcommand{\I}{\item}
\newcommand{\f}{\frame}
\newcommand{\ft}{\frametitle}

\title{Introduction and Overview}
\subtitle{for the FDC Mini-Review}
\author[M.\ Ito]{Mark M.\ Ito}
\date{February 1, 2008}
\institute[JLab]{Jefferson Lab}

\begin{document}

\f{\titlepage}

\f{
  \ft{Outline}
  \bi
  \I physics goals
  \I GlueX detector
  \I reviews: past, present and future
  \I general design considerations
  \I backgrounds
  \I specifications
  \I resolution estimates
  \I conclusions
  \ei
}

\f{
  \centerline{\Large physics goals}
}
\f{
  \ft{goals and requuirements}

The search for exotic mesons is at a turning point. The experiments at
BNL, Protvino, and at LEAR which have reported evidence for exotic
mesons have terminated data taking; data analysis is completed...new
experiments are ahead of us, COMPASS at CERN and BESIII in the
immediate future, the Hall-D experiment at the upgraded JLab
facility and PANDA at GSI in the medium-range future. --- Eberhard
Klempt

  \bi
  \I Map the spectrum of hybrid mesons (gluonic excitations)
    \bi
    \I Start with those with exotic quantum numbers
    \I Make contact with spectroscopy of non-exotic states
    \ei

  \I Analysis will require
    \bi
    \I partial-wave analysis
    \I identification of exclusive final states
    \I detailed understanding of backgrounds
    \I large event samples
    \I confirmation of states in multiple decay channels
    \ei
  \ei
}
\f{
  \ft{lattice mass predictions}
  Lowest mass expected to be $\pi_1(1^{−+})$ at $1.9 \pm 0.2$~GeV
$$
  \includegraphics[height=3in]{hybrids_lattice.png}
$$
}
\f{
  \ft{representative kinematics}
$$
  \includegraphics[height=3in]{kinematics.png}
$$
}
\f{
  \ft{anything else?}
  \bi
  \I Hall D a unique facility
    \bi
    \I high-energy tagged photon beam
    \I coherent bremsstrahlung $\Rightarrow$ nearly mono-energetic beam
    \I linear polarization
    \ei
  \I driven to a general purpose detector
    \bi
    \I charged and neutral particle detection
    \I very large acceptance
    \ei
  \I other physics possible!
    \bi
    \I come to the workshop, PHP2008, March 6-8, at JLab
    \I ``Photon-hadron physics with the GlueX detector at Jefferson Lab''
    \ei
  \ei
}
\f{
  \centerline{\Large GlueX detector}
}
\f{
  \ft{the detector}
  $$
  \includegraphics[height=3in]{gluex_detector_3d.png}
  $$
}
\f{
  \ft{elevation view}
  $$
  \includegraphics[height=3in]{gluex_detector.png}
  $$
}
\f{
  \centerline{\Large reviews: past, present and future}
}
\f{
  \ft{review roster}
  \bi
  \I \textcolor{red}{Drift Chamber Review, Hall B and D, March 2007}
  \I CD-2 12 GeV Project Review, June 2007
  \I
  \includegraphics[height=0.5in]{howard_pwc.png}\textcolor{red}{$\leftarrow$FDC
    Mini-Review, February
    2008$\rightarrow$}\includegraphics[height=0.5in]{Larry_Weinstein.jpg}
  \I \textcolor{red}{Drift Chamber Review, March 2008}
  \I CD-3 Project Review, June 2008
  \ei
}
\f{
  \ft{Drift Chamber Review, Hall B and D, March 2007}

  Recommendations (Hall D)

  \begin{enumerate}

    \I Priority should be given to studying design modifications that
    would significantly reduce the amount of material in the GlueX
    tracking chambers.

    \I Additional resources and expertise should be applied to the
    development of track reconstruction software for GlueX, and to a
    complete and realistic hit-level simulation of the GlueX
    spectrometer.

  \end{enumerate}
}
\f{
  \centerline{\Large general design considerations}
}
\f{
  \ft{really general stuff}
  \bi
  \I characteristics of the acceptance
    \bi
    \I must be large
    \I must be smooth
    \I must be well understood
    \I driven by need to do a partial-wave analysis
    \ei
  \I resolution counts
    \bi
    \I needed for missing mass identification
    \I reduces combinatoric background under narrow resonances
    \I sharpens results from the partial-wave analysis
    \I surprises, unintended benfits, etc.
    \ei
  \I robust pattern recognition
    \bi
    \I robustness $\Leftrightarrow$ simplicity $\Leftrightarrow$
    speed of reconstruction
    \I implications for
      \bi
      \I online level 3 trigger
      \I offline processing speed
      \ei
    \ei
  \ei
}
\f{
  \ft{$\forall$ FDC's}
  \bi
  \I drift chambers
  \I 4 packages
  \I pipeline readout
  \I variations on design work already done for the reference design
  \ei
}
\f{
  \ft{design options}
  \be
  \I two cathode strip planes per wire plane (reference)
    \bi
    \I six anode wire planes per package
    \I twelve cathode strip planes
    \I cathodes at $\pm 75^\circ$
    \I each anode plane rotated $60^\circ$ from previous anode plane
    \I TDC readout for anodes
    \I FADC readout for cathodes
    \ei
  \I no cathode strips (wires only)
    \bi
    \I twelve(?) anode wire planes per package
    \I orientation options
      \be
      \I each plane rotated $60^\circ$ from previous
      \I pairs of planes with half-cell offset, pairs rotated
      $60^\circ$ from previous
      \ee
    \ei
  \I one cathode strip plane per anode wire plane (hybrid)
  \ee
}
\f{
  \centerline{\Large backgrounds}
}
\f{
  \centerline{\Large specifications}
}
\f{
  \centerline{\Large resolution estimates}
}
\f{
  \centerline{\Large conclusions}
}

\f{
\ft{leftovers}
momentum and angle range
position resolution
dominated by multiple scattering
material reduction
wires-only
nominal 2 cathode-readout layers per anode layer design
savings in electronics
pipeline data-acquisition
fadc readout combines time and charge
dedx possible
cost savings
with cathode and anode read-out: disambiguate using timing
  possible to get 3-d points in a single tracking layer
time resolution
no segmentation in azimuth
  cdc
  clas
  tpc
narrow resonances: good resolution reduced background
robust pattern recognition
full court press
backgrounds
  electronic noise
  e\&m
  hadronic
  combinatoric
}

\end{document}

%%% end of latex file %%%%
