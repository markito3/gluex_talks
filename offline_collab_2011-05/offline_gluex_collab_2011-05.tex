# February 9, 2011 

\ft{JLab IT in the 12 GeV Era}
\bi
\I Review May 20 at JLab
\I internal to JLab
\I all Halls, DAQ, infrastructure
\I "dry run" for external review next year
\I draft computing estimate circulated
\I Matt Shepherd on the review committee
\I Grid posture: opportunistic usage on demand
\ei


\I Reconstruction Parameters vs. Calibration Constants
\bi
\I David gave a talk on the JANA configuration scheme
\I parameters: effect algorithmic treatment of data -> in configuration file with sensible defaults, managed by source code version control (Subversion)
\I calibration constants: run-dependent corrections to raw measured quantities -> from database, may have multiple interations
\I Instituted frozen versions of the calibration constants
\I work needs to be done to disentangle the two
\ei

\I Compiler Warnings: many fewer

\I Use of Profiling Tools to Guide Optimization: talk from Dmitry

\I Timing Resolution

\bi
\I Kei gave several presentations
\I original problem: using a "T0" coming from a smeared start counter time order 100's of picoseconds
\I plan: use RF bucket timing, perfect resolution on our scale
\I solution: add place holder data member to DVertex class
\ei

# February 23, 2011

\ft{Event Display}

\bi
\I Issue use
\bi
\I bCNU (Java, Hall B)
\I EVE (Root, several HEP collaborations)
\ei
\I Geometry
\I Support
\I Features
\ei

\ft{Calibration Database Update}

\bi
\I Highly developed code set
\I Many presentations by Dmitry
\I No release as of this date
\I Remote collaboration has not progressed as hoped
\ei

# March 9, 2011

\ft{Single-Instruction-Multiple-Data Status}

\bi
\I Changes introduced at end of last year by Simon
\I SIMD-off option introduced in January and made the default 
\I streaming SIMD extensions (SSE) are supported depending on the particular CPU architecture
\I David did a study showing a 5-10\% speed increase for multi-track events
\I Richard worked on a scheme for sensing the SSE capability of the CPU and setting appropriate variables that could be used by the make system
\I SIMD-off remains the default
\ei

\ft{Offline Computing Requirements and the Grid}

\bi
\I David organized a set of meetings to discuss what role the grid should be playing in our offline computing plans, if any
\I consensus around a robust storage resource manager (SRM) capability at JLab
 to feed off-site grid-based clusters.
\ei

# March 23, 2011

\ft{Reorganization of Documentation}

\bi
\I Zisis suggested a re-org, especially the "getting started" stuff.
\I Beni has extensively re-arranged the top level offline software wiki page.
\I Mark collected the info on his build tools in a common location.
\ei

\ft{Floating point entropy causing segfaults}

\bi
\I issue with using computed floating point values in a user-supplied Standard Template Library comparison function on 32-bit operating systems when the compiler uses the x87 floating point co-processor
\I seg will fault
\I David reviewed our code and removed as such instances.
\I Not a proof that there are not other things lurking.
\ei

\ft{Make system, includes, in JANA and sim-recon}

\bi
\I Issue with checked out (and possibly modified) versions of include files and those used from an installed location (like a public build)
\I Matt proposes reliance exclusively on the checked out versions
\I This is on the task list
\ei

# April 6, 2011

\ft{&chi;<sup>2</sup>'s of tracking}

\I Kei has done a studies of the  &chi;<sup>2</sup> values returned from the tracking code
\I He finds departures from the ideal distribution; in terms of &chi;<sup>2</sup> probability, there are peaks at both high and low probability.
\I We need work on tracking errors.

# April 20, 2011

\ft{Definition of primaries}

\I Beni introduced change to associate hits with the "true" particles that produced them, especially for the tracking chambers. Useful studies have resulted from this change
\I "marked" all particles as primary (with exceptions)
\I caused a problem for others when the distinction between primary and secondary particles no longer obvious
\I Will have a meeting to write hit-particle correlations to HDDM output from HDGeant.

\ft{Final Thoughts}

\bi
\I "standard" terminology vs. "as coded" terminology
\I code review needed
\I documentation requirements easily ignored
\ei
