\documentclass[xcolor=dvipsnames,hyperref={pdfpagelabels=false}]{beamer}

\usetheme{Boadilla}

\newcommand{\bi}{\begin{itemize}}
\newcommand{\ei}{\end{itemize}}
\newcommand{\be}{\begin{enumerate}}
\newcommand{\ee}{\end{enumerate}}
\newcommand{\bc}{\begin{center}}
\newcommand{\ec}{\end{center}}
\newcommand{\bd}{\begin{description}}
\newcommand{\ed}{\end{description}}
\newcommand{\I}{\item}
\newcommand{\f}{\frame}
\newcommand{\ft}{\frametitle}

\title{Offline Software Overview}
\subtitle{GlueX Collaboration Meeting}
\author[Mark Ito]{Mark M.\ Ito}
\date{October 3, 2014}
\institute[JLab]{Jefferson Lab}

\begin{document}

\f{\titlepage}

\f{
\bi
\I David produced new field maps
\I Sean has the backend database code for EventStore all done
\I David changed CCDB so that one-dimensional arrays treated the same whether they are single column or single row
\ei
}

\f{
\bi
\I Simon has changed the code to get the parameter for the minimum number of hits on a track candidate from a JANA configuration parameter.
\I David brought us up-to-date on the status of doing reconstruction with EVIO data derived from simulation output.
\I Will has observed a late tail in the BCAL time distributions for pions, and not seen for photons.
\ei
}

\f{
\bi
\I The solution is to use the MD5 check-sum that is generated and downloaded along with each resource file to check file integrity each time the file is read. David is working on implementation for the next version of JANA. 
\I Haswell CPU Testing: David reported on tests he has done bench-marking a demo machine that SciComp had on loan for this purpose. He compared the demo Haswell machine with an Ivy Bridge gluon machine in the Counting House. 
\I Paul has prompted Mark to take up work again on a global build system for GlueX. Paul would like to see multiple versions of each package co-exist in the tree with dependencies among them taken care of. 
\I Richard requested that someone convert the build of hdds from GNUMake to scons. David agreed to look into what is involved. 
\ei
}

\f{
\ft{Tagger Reconstruction and Global Event Timing}
\bi
\I Richard led us through his proposal for introducing a random global time offset to all events, consistent with the 500 MHz RF time structure. This would simulate the real-life uncertainty due to event-to-event trigger latency variations and the intrinsic jitter due having a fully pipelined data acquisition system driven on a 250 MHz clock. At present, all events are analyzed as if the true RF bucket is known, a priori. Also the true beam photon energy is also assumed in the analysis. 
\I HDDM Calls: Conversion to C++ API: All invocations of the HDDM API have been upgraded to use the C++ version rather than the old C version. This change comes in many places in the sim-recon tree. The exception is the GEANT-3-based code, in particular the current HDGeant; that conversion will come with the conversion to Geant4.
\I HDDM Integrity Checks: The HDDM library now supports "integrity checks" where a cyclic redundancy check (CRC) code is computed event-by-event and can be written with the event. Downstream programs can check that incoming data has not been corrupted.
\I HDDM\_s Changes: The template for simulation output has been changed to rationalize the arrangement of data in two areas: (1) the mix of truth and hit information and (2) the detector component hierarchy. A generic example of the latter is using enclosing elements to indicate "layer" rather than encoding "layer" as an attribute of each hit. Note that changes of this nature are indicated in the HDDM\_s version in the template contained in every data file.
\I Tagger Analysis: CCDB tagger tables have been rationalized and new C++ classes introduced in sim-recon to accommodate the new scheme. The start counter was used as a model for the low-level hit classes in the DAQ plug-in. There were some modifications needed to HDGeant (GEANT3) code to write the new classes.
\I Analysis Library Changes: Paul will need to change the analysis library to deal with the ambiguity of the "correct" tagger hit. For now he will use the factory tag for the tagger to get the true tagger hit. That makes things more or less as things were before. Once the merge onto the trunk is done, he will implement the necessary changes.
\ei
}

\f{
\bi
\I gxtwist: Gxtwist is a stand-alone simulation of the tagger hall. Richard has updated the geometry based on the latest drawings. This has been checked in on the trunk. He has also added reality to the radiator with separate positions for the diamond and the amorphous radiators. 
\I Re-Creating Tracks: There was a discussion on the email list about re-creating tracks with mass hypotheses that are not present in the set coming from track reconstruction. This can be very time-consuming. In these cases usually the fit failed when using the mass assumption in question. Simon is working on solving this problem in tracking so a full set of mass hypotheses are available in the REST file. See the email discussion for details. 
\I Comparing Simulated HDDM and EVIO Files: Sean has done comparisons between hit information between simulation native output (HDDM format) and EVIO data derived from same using the new trunk from Richard.
\ei
}

\f{
\ft{Other Talks}
\bi
\I Offline Data Quality Monitoring: see Paul Mattione's talk
\I Simulating the Commissioning Detector Configuration: see Sean Dobb's talk
\I Analysis of Six Final States Using Data Challenge 2 MC: see Ryan Mitchell's talk (Physics Session)
\ei
}

\f{
\ft{Data Challenge 3}
\bi
\I Original goals:
  \be
  \I process fake raw data resident in the tape library
  \I render the data in EVIO format
  \I perform more efficient generation of the electromagnetic background
  \I use the Geant4 version of HDGeant
  \I incorporate new code for obtaining constants from the CCDB
  \ee
\I Dropped \#3, \#4, \#5 during summer
\I Dropped \#1, \#2 in September
\I Other large scale simulation tasks superceded: commissioning geometry, tagger hall
\ei
}

%\f{
%\input a.tex
%}

\end{document}
