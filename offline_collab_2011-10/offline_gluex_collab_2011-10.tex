\documentclass[xcolor=dvipsnames]{beamer}

\usetheme{Boadilla}

\newcommand{\bi}{\begin{itemize}}
\newcommand{\ei}{\end{itemize}}
\newcommand{\be}{\begin{enumerate}}
\newcommand{\ee}{\end{enumerate}}
\newcommand{\I}{\item}
\newcommand{\f}{\frame}
\newcommand{\ft}{\frametitle}

\title{Status of the Offline Software}
\subtitle{GlueX Collaboration Meeting}
\author[M.\ Ito]{Mark M.\ Ito}
\date{October 6, 2011}
\institute[JLab]{Jefferson Lab}

\begin{document}

\f{\titlepage}

\f{

\ft{Review: JLab IT in the 12 GeV Era}
\bi
\I review May 20 at JLab
\I internal to JLab
\I ``dry run" for external review next year
\I return to this later in talk
\ei

}

\f{
\ft{Correlating Raw Hits and Generated Particles}

\bi
\I Beni put in a system to correlate raw hits with generated particles.
\I Some adjustments were necessary to distinguish primary and secondary particles.
\I Used for studies, in particular Nathan's cascade events
\ei

}

\f{
\ft{IT Review}

\bi
\I Review was held May 20, 2011
\I Goal: get a good understanding of progress towards IT in the 12
GeV era, and discover if there are areas that might need increased
effort in the coming year
\I Reviewed: DAQ, Halls A,B,C,D, Accelerator (Controls and Physics), Scientific Computing and LQCD, Management Information Systems (MIS), Computer Networking \& Infrastructure (CNI)
\I Hall D (MMI) presented:
  \I Requirements: data rate, reconstruction time, MC needs, storage
  \I Status of the Software: geometry, simulation, reconstruction, calibration database
  \I Planning, Tests, System Engineering: project managements, code testing, source code control, documentation, communications, off-site computing use
\I Committee report (privately obtained):
  \I General
    \I clear...that there is an enourmous amount of work in progress...no critial showstoppers at this stage.
    \I Consider: designate a person to be in charge of the big picture
    \I Consider: identify pieces of DAQ and analysis chain...identify those in charge...demonstrate a plan for integration ans stress testing of the entire...chain.
    \I Recommend: establish a more formal joint effort to ensure analysis software is ready... [among the Halls?]
    \I Recommend: establish dates for stress tests
  \I Hall D
    \I Finding: 2 FTE of software effort on staff...might...represent insufficient manpower
  \I Finding: computing requirements should be firmed up...include details on...offsite computing
  \I Finding: cost of tape...a very low figure...provides backup...
  \I Recommendation: progress...should be carefully watched...more staffing at JLab should be added to software development...firm commitments with MOU's from ...the user community should be obtained in the coming year.
  \I Recommendataion: Details for a significant off-site simulation should be fleshed out...letters of intent obtained...formal MOU's...obtained later...
  \I Recommendatation: keep a duplicate of all raw data
\I Other Items
  \I some really good recommendations produced
  \I summer of next year for full review
  \I focus likely more on the Halls
  \I why is IT Division reviewing Physics and Accelerator Division?

\ei

}

\f{
\ft{DFCALPhoton vs.\ DFCALShower}

\bi
\I Mihajlo: new names for some of the calorimeter classes.
\I The FCAL clases have been changed to be consistent with the naming convention used for the BCAL
\I "clusters" and "showers"
\ei

}

\f{
\ft{b1pi Fix}

\bi
\I reconstruction efficiency for the weekly automatic b1pi analysis started to show only 5\%
\I was around 50\%
\I problem was found and fixed by David
\ei

}

\f{
\ft{Grid Make System}

\bi
\I Richard describe a system he is developing to
  \bi
  \I automatically install and build the GlueX software suite on an arbitrary Unix-based system and
  \I use the build to create user-specified output, the result of either simulation, reconstruction, and/or analysis.
  \ei
\ei

}

\f{
\ft{Offline Coordinator Election}
\bi
\I MMI ran unopposed
\I won
\I two more years
\ei
}

\f{
\ft{Doxygen Documentation}

\bi
\I David has given it a makeover
\I Suggests minimum standards of in-file documentation
\ei

}

\f{
\ft{TOF reconstruction}

\bi
\I Reviewed by Simon
\I No time-walk correction algorithm implemented
\I Cases where track crosses two adjacent paddles in a view not explicitly treated
\ei

}

\f{
\ft{Calibration Database}

\bi
\I Alpha release mid-July
\I Beta release, matter of days
\I Conversion of files to database representation solved
\I CLAS12 is planning to use it
\I Concept for large files (magnetic field maps) being developed
\I See Dmitry's talk in this session
\ei

}

\f{
\ft{Geant4 Tutorial}

\bi
\I Team from SLAC, last here in 2006
\I Comprehensive hand-on tutorial workshop
\I Proposal submitted to JSA Initiatives Fund
\I Summer 2012
\ei

}

\f{
\ft{New data format plug-ins}

\bi
\I Elliott created two new plug-ins 
\I {\bf danaevio plug-in}. This plug-in takes raw hit objects and reconstructed objects and generates an EVIO tree
\I {\bf rawevent plug-in}.  Plug-in to take raw EVIO data and turn it into DANA objects
\ei

}

\f{
\ft{Non-reproducible event reconstruction}

\bi
\I Reported by Will
\I changing four-vectors, particle roster
\I Found and fixed by David
\I Another 1\% PID problem found and fixed by Paul
\ei

}

\f{
\ft{Restructuring particle classes}

\bi
\I Paul institued new structure of top-level reconstructed particle classes
\I more intuitive
\I reconstructs neutrons
\ei

}

\f{
\ft{Cyber Event}

\bi
\I Entrance point: webservers
\I Exploit technical, not social engineering
\I Many machines compromised, mainly Windows
\I data "exfiltrated"
\I weeks between initial intrusion and data shipment
\I IT Division was extremely cautious about re-opening services
\I New personal web partition introduced
\I Attack apparently related to other intrusions elsewhere
\ei

}

\f{
\ft{Translation Tables}

\bi
\I Elliott presented plan on storing the correspondences between electronic channel addresses and named detector elements.
\ei

}

\f{
\ft{Automated b1pi jobs}

\bi
\I semi-weekly jobs now running
\I Beni added some more histograms, user-friendly web presentation
\ei

}

\f{
\ft{Environment management and directory structure}

\bi
\I Maurizio Ungaro: new JLab-wide-oriented scheme for environment variable set-up and builds of major software packages
\I Sensible defaults provided
\I Augmented with rpm's and deb's which he maintains (for off-site especially)
\ei

}

\f{
\ft{Object-relational mapping (ORM) and an online configuration database}

\bi
\I Dmitry: introduction to ORM's
\I working with Sascha Somov on implementation for electronics crate configuration
\ei

}

\f{
\ft{DST data format}

\bi
\I Richard and Igor: production of a DST to facilitate grid computation
\I presented plan on what to keep and what to throw out
\I work in progress
\ei

}

\f{
\ft{Tracking dE/dx update}

\bi
\I Simon discovered a units problem
\I fixed now
\ei

}

\f{
\ft{BCAL reconstruction update}

\bi
\I problems observed in errors when doing kinematic fitting
\I see's better behavior with Matt's new reconstruction algorithm
\I Will Levine working on the new 1-2-3-4 segmentation of the BCAL
\I will implement in KLOE algorithm
\ei

}

\f{
\ft{Reconstruction chi-squared, particle ID}

\bi
\I Paul studying errors from various detector systems
\I Integrating information into particle ID scheme
\I See his talk in this session
\ei

}

\f{
\ft{Reconstruction Sub-Groups}

\bi
\I Matt proposed forming subgroups focused on specific areas
\I Reports will be given at every meeting
\I Chairpeople were appointed
  \bi
  \I Tracking: Simon
  \I PID: Paul M.
  \I Photons: Matt
  \ei
\I A list of topics to be addressed has been generated by collaboration members
\ei

}

\f{
\ft{Plug-in to extract single particles from multi-particle events}

\bi
\I David wrote a new plug-in that recasts one multi-particle into one (or more) single particle event(s).
\I allows single particle studies with spectra found in multi-particle events
\ei

}

\f{
\ft{Tracking explained}

\bi
\I Simon gave us a summary talk
\I And produced a new document
\ei

}

\f{

\ft{Final Thoughts}

Good News

\bi
\I Event reconstruction from hit-level Monte Carlo is routine.
\I Multi-threading technology is established; essential feature for the future.
\ei

To Be Worked On

\bi
\I ``standard'' terminology vs.\ ``as coded" terminology $\Rightarrow$ confusion
\bi
\I code review needed?
\I documentation requirements easily ignored
\ei
\I project management system still needed
\I paradigm of ``ready to process on day one'' not established
\ei
}

\end{document}

%%% end of latex file %%%%
