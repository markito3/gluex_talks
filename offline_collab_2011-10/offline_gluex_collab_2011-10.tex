\documentclass[xcolor=dvipsnames]{beamer}

\usetheme{Boadilla}

\newcommand{\bi}{\begin{itemize}}
\newcommand{\ei}{\end{itemize}}
\newcommand{\be}{\begin{enumerate}}
\newcommand{\ee}{\end{enumerate}}
\newcommand{\I}{\item}
\newcommand{\f}{\frame}
\newcommand{\ft}{\frametitle}

\title{Status of the Offline Software}
\subtitle{GlueX Collaboration Meeting}
\author[M.\ Ito]{Mark M.\ Ito}
\date{May 9, 2011}
\institute[JLab]{Jefferson Lab}

\begin{document}

\f{\titlepage}

% February 9, 2011 

\f{

\ft{Review: JLab IT in the 12 GeV Era}
\bi
\I review May 20 at JLab
\I internal to JLab
\I all Halls, DAQ, infrastructure
\I ``dry run" for external review next year
\I draft computing estimate circulated
\I Matt Shepherd on the review committee
\I grid posture: opportunistic usage on demand
\ei

}


\f{
\ft{Reconstruction Parameters vs. Calibration Constants}
\bi
\I {\bf parameters}: effect algorithmic treatment of data $\rightarrow$ in configuration file with sensible defaults, managed by source code version control (Subversion)
\I {\bf calibration constants}: run-dependent corrections to raw measured quantities $\rightarrow$ from database, may have multiple interations
\I David gave a talk on the JANA configuration scheme
\I instituted frozen versions of the calibration constants as interim measure
\I work needs to be done to disentangle the two
\ei
}

\f{
\ft{Timing Resolution}

\bi
\I Kei gave several presentations
\I original problem: using a ``T0" coming from a smeared start counter time with resolution of order hundreds of picoseconds
\I plan: use RF bucket timing, perfect resolution on our scale
\I solution: add place holder data member for start time to DVertex class, set to zero for now
\ei
}

% February 23, 2011

\f{
\ft{Event Display}

\bi
\I Issue: choice of using
\bi
\I bCNU (Java, Hall B)
\I EVE (Root, several HEP collaborations)
\ei
\I Geometry
\I Support
\I Features
\I Manpower
\ei
}

\f{
\ft{Calibration Database}

\bi
\I Highly developed code set
\I Many presentations by Dmitry
\I No release as of this date
\I Remote collaboration has not progressed as hoped
\ei
}

% March 9, 2011

\f{
\ft{Single-Instruction-Multiple-Data Status}

\bi
\I Changes introduced at end of last year by Simon
\I SIMD-off option introduced in January and made the default 
\I streaming SIMD extensions (SSE) are supported depending on the particular CPU architecture
\I David did a study showing a 5-10\% speed increase for multi-track events
\I Richard worked on a scheme for sensing the SSE capability of the CPU and setting appropriate variables that could be used by the make system
\I SIMD-off remains the default
\ei
}

\f{
\ft{Offline Computing Requirements and the Grid}

\bi
\I David organized a set of meetings to discuss what role the grid should be playing in our offline computing plans, if any.
\I Consensus around a robust storage resource manager (SRM) capability at JLab
 to feed off-site grid-based clusters.
\ei
}

% March 23, 2011

\f{
\ft{Reorganization of Documentation}

\bi
\I Zisis suggested a re-org, especially the "getting started" stuff.
\I Beni has extensively re-arranged the top-level offline software wiki page.
\I Mark collected the info on his build tools in a common location.
\ei
}

\f{
\ft{Floating point entropy causing segfaults}

\bi
\I issue with using computed floating point values in a user-supplied Standard Template Library comparison function on 32-bit operating systems when the compiler uses the x87 floating point co-processor
\I seg will fault
\I David reviewed our code and removed such instances.
\I Not a proof that there are not other things lurking.
\ei
}

\f{
\ft{Make system, includes, in JANA and sim-recon}

\bi
\I issue with checked out (and possibly modified) versions of include files in conflict with those used from an installed location (like a public build)
\I Matt proposes reliance exclusively on the checked out versions
\I this is on the task list
\ei
}

% April 6, 2011

\f{
\ft{$\chi^2$ of tracking}

\bi
\I Kei has done studies of the  $\chi^2$ values returned from the tracking code.
\I He finds departures from the ideal distribution; in terms of $\chi^2$ probability, there are peaks at both high and low probability.
\I We need work on tracking errors.
\ei
}

% April 20, 2011

\f{
\ft{Definition of primaries}

\bi
\I Beni introduced change to associate hits with the ``true" particles that produced them, especially for the tracking chambers. Useful studies have resulted from this change.
\I method: ``marked" all particles as primary (with exceptions)
\I caused a problem for others when the distinction between primary and secondary particles blurred
\I will have a meeting to write hit-particle correlations to HDDM output from HDGeant.
\ei
}

\f{

\ft{Final Thoughts}

Good News

\bi
\I Event reconstruction from hit-level Monte Carlo is routine.
\I Multi-threading technology is established; essential feature for the future.
\ei

To Be Worked On

\bi
\I ``standard'' terminology vs.\ ``as coded" terminology $\Rightarrow$ confusion
\bi
\I code review needed?
\I documentation requirements easily ignored
\ei
\I project management system still needed
\I paradigm of ``ready to process on day one'' not established
\ei
}

\end{document}

%%% end of latex file %%%%
