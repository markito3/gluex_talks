\documentclass{beamer}

\usepackage{epsfig}
\usetheme{Boadilla}

\title{Tracking Software Thoughts}
\author[M.\ Ito]{Mark M.\ Ito}
\date{December 12, 2007}
\institute[JLab]{Jefferson Lab}

\newcommand{\bitem}{\begin{itemize}}
\newcommand{\eitem}{\end{itemize}}
\newcommand{\benum}{\begin{enumerate}}
\newcommand{\eenum}{\end{enumerate}}
\newcommand{\bcent}{\begin{center}}
\newcommand{\ecent}{\end{center}}

\begin{document}

\frame{
\frametitle{tracking software thoughts}
\bitem
\item what i've been working on
  \bitem
  \item trajectory class
  \item fitter class
  \eitem
\item what our current goals should be
\item what we need to do for the technology review
\item what we need to do for the drift chamber review
\eitem
}
\frame{
  \frametitle{trajectory class}
    \bitem
    \item swimming (straight, helix, general B)
    \item distance-of-closest-approach services
      \bitem
      \item to a line
      \item to a point
      \item et cetera
      \eitem     
    \item energy loss (not started)
      \bitem
      \item read in HDDS
      \item put in dE/dx
      \eitem
    \eitem
}
\frame{
  \frametitle {fitter class}
    \bitem
    \item takes as input:
      \bitem
      \item trajectory generator
      \item residual function
      \eitem
    \item uses GNU Scientific library
    \item uses Levenberg-Marquardt algorithm
    \item works on general set of statistically independent residuals
      \bitem
      \item pseudo-points
      \item drift distances
      \item cathode center-of-gravity's
      \eitem
    \item works with general trajectory (in progress)
    \eitem
}
\frame {
\frametitle{what our current goals should be}
  \bitem
  \item robust general track finder and fitter
    \bitem
    \item good efficiency
    \item good resolution
    \item not necessarily great on either, may not include
      \bitem
      \item correlations of hits due to multiple scattering
      \item microscopic energy loss correction
      \eitem
    \eitem
  \item usable for studies of other detectors
  \item usable for studying effect of background
  \item usable for studying efficiencies of signals
  \item combine CDC and FDC
  \item fitter should be quasi-independent of chamber geometry/configuration
  \item track finder necessarily geometry dependent
  \eitem
}
\frame{
\frametitle{what we need to do for technology review}
  \bitem
  \item FDC design well-advanced
  \item tests on this technology have been done
  \item that cathodes help with pattern recognition should be stipulated
  \item need to find a reason it won't work at this point
    \bitem
    \item too much multiple scattering?
    \item Lorentz effect correction intractable?
    \eitem
  \item variations on current theme are do-able
  \item what if we make the ``wrong'' technology decision?
    \bitem
    \item change request post-CD3?
    \item build it and live with it?
    \eitem
  \eitem
}
\frame{
\frametitle{what we need to do for the drift chamber review}
  \bitem
  \item robust tracker as mentioned above under goals
  \item demonstrates a complete handle on the fundamentals of the problem
  \item enables better context for other studies
  \eitem
}

\end{document}

%%% end of latex file %%%%
