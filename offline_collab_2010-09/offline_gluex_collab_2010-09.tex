\documentclass[xcolor=dvipsnames]{beamer}

\usetheme{Boadilla}

\newcommand{\bi}{\begin{itemize}}
\newcommand{\ei}{\end{itemize}}
\newcommand{\be}{\begin{enumerate}}
\newcommand{\ee}{\end{enumerate}}
\newcommand{\I}{\item}
\newcommand{\f}{\frame}
\newcommand{\ft}{\frametitle}

\title{Status of the Offline Software}
\subtitle{GlueX Collaboration Meeting}
\author[M.\ Ito]{Mark M.\ Ito}
\date{September 10, 2010}
\institute[JLab]{Jefferson Lab}

\begin{document}

\f{\titlepage}

% May 18, 2010 

\f{

\ft{Progress 1}

\bi
\I {\bf Writing out tracking results}. Implemented. HOWTO save tracking results to an HDDM or EVIO file for later playback written.\cite{save-tracking} (David)
\I {\bf Offline Computing change request} has been entered in the official project software. (Mark, Eugene)
\I {\bf GlueX Software Workshop}. Theme: parallel computing. Subcommittee formed David, Beni, Simon, and Sascha, Ed Brash, and Ole Hansen. To be held in December or January. Proposal to the JSA Initiative committee submitted. (David)
\I {\bf GlueX Computing Plan}. Effort started to revise estimates of resource requirements. Early stages. (Mark)
\I {\bf Custom matrix classes} to speed up tracking.\cite{Simon-on-tracking}{Simon's talk} (Simon)

% June 1, 2010

\I {\bf Warning-Free code}. New policy.
%#* David mentioned that the next step would be to create histograms. Right now a root tree is produced but the only information available is that the reconstruction ran to completion. He has a script that will produce histograms and mail out the results. He and Mark will create an example that runs on the cron job output.
\I {\bf HOWTO do a kinematic fit for etapi0p events} written\cite{kinematic} (Blake)

\I {\bf HOWTO use a pre-built release} writen (Mark)\cite{pre-built}

\I {\bf Effect of a more uniform field}. Studies done with various fields.\cite{uniform-field} (David, Simon)

% June 15, 2010

\I {\bf Sim-Recon Tagged Releases}. New wiki page.\cite{sim-recon-wiki} (Mark)

\I {\bf How HDGeant defines
  time-zero for physics events} new wiki page\cite{time-zero} (Richard)

% Bug/issue tracking software==

\I {\bf Test of JANA on a 48-core machine} Scaling of performance observed. Some questions remain.\cite{multi-core} (David)

\I {\bf Standard math library}. Things like matrices, vectors. Not needed yet. (all)

% June 29, 2010

\I {\bf JANA 0.6.2} A new version of JANA.\cite{new-jana} (David)

\I {\bf Calibration and Conditions Database specification} Document exists\cite{ccdb} (David, Mark)
\I {\bf Installation of Grid Client Tools at JLab} Agreement in principle. Still need to get Globus ports opened.\cite{grid-tools} (Richard, Mark)

\I {\bf Grid certificates being granted} (Sandy Philpott)

\I {\bf Structure of BCAL Smearing}. A reworking, moving the code from the reconstruction library to the smearing program. (Dan Bennett)

\I {\bf Issue tracking code deployed} MantisBT was chosen. Offline work being tracked with it now.\cite{mantis} (Elliott)

% July 13, 2010

\I {Secrets of JANA plugins revealed} Nice talk.\cite{jana-plugin} (David)

% July 27, 2010

\I {\bf Event Display}. See Elliott's talk. There is a HOWTO, ''ded: Install \& Run''\cite{ded-doc}(Andrew Blackburn)

\I {\bf Generalized particle gun}. Simple perl-based approach to a script driven single particle gun.\cite{perl-particles} (Mark)

\I {\bf Documentation Policy} Written down, captures current practices, introduced FAQ\cite{doc-policy} (Mark)

% August 10, 2010

\I {\bf b1pi analysis in cron job}. Plots that are now produced automatically, weekly basis.\cite{b1pi-auto} (David, Mark)

\I {\bf Hall-D File Formats} Lots of discussion recently.\cite{halld-file-format} (all)

% August 24, 2010

\I HDGEANT auto-smearing. New feature.\cite{auto-smear} (David)

\ei

}

\f{
\ft{Summary and Remarks\footnote{copied from last time}}
\bi
\I Progress in many areas
\I Major current challenges
\be
\I Extraneous clusters from hadronic interactions
\I Speed of charged particle tracking
\bi
\I Technological fix?
\I Algorithmic fix?
\ei
\ee
\I Minor current challenges
\be
\I Output format? HDDM? EVIO? ROOT? All of the above?
\I Global timing and PID
\I Doing a native build on 64-bit OS's (e.\ g., Snow Leopard, Linux x86\_64)
\ee
\I Long-term challenges: see the Prioritized Task List
\I Need for better documentation
\ei
}

\f{
\ft{References}
\begin{thebibliography}{99}
\tiny
\bibitem{save-tracking}$[[$HOWTO save tracking results to an HDDM or EVIO file for later playback$]]$
\bibitem{Simon-on-tracking} reference needed

\end{thebibliography}

}

\end{document}



\bibitem{kinematic}[[HOWTO do a kinematic fit for etapi0p events]]
\bibitem{pre-built}[[HOWTO use a pre-built release]]
\bibitem{uniform-field} ???
\bibitem{sim-recon-wiki}[[Sim-Recon Tagged Releases]]
\bibitem{[[How HDGeant defines time-zero for physics events]].}
\bibitem{multi-core}{[[Media:20100615_48cores.pdf]]}
\bibitem{new-jana}https://mailman.jlab.org/pipermail/halld-offline/2010-June/000314.html
\bibitem{ccdb} ???
\bibitem{grid-tools}{[http://markito3.wordpress.com/2010/06/23/notes-on-grid-client-tools-meeting/], http://zeus.phys.uconn.edu/UConn-OSG/client-services.html
\bibitem{bcal-smear}{http://argus.phys.uregina.ca/gluex/DocDB/0015/001552/001/BCAL_Software_progress_062910.pdf
\bibitem{mantis}{https://halldnew.jlab.org/mantisbt/}
\bibitem{jana-plugin}[[Media:20100713_plugins.pdf]]
\bibitem{ded-doc}[[HOWTO ded: Install & Run]]
\bibitem{perl-particles}https://mailman.jlab.org/pipermail/halld-offline/2010-July/000330.html
\bibitem{doc-policy}[[Software Documentation Structure]] and [[GlueX Offline FAQ]]
\bibitem{b1pi-auto}???
\bibitem{halld-file-format}[[GlueX Offline Meeting, August 10, 2010]] and [[GlueX Offline Meeting, August 24, 2010]]
\bibitem{auto-smear}https://mailman.jlab.org/pipermail/halld-offline/2010-August/000348.html
\end{thebibliography}
}

\end{document}

%%% end of latex file %%%%
