\documentclass{beamer}

\usetheme{Boadilla}

\newcommand{\bi}{\begin{itemize}}
\newcommand{\ei}{\end{itemize}}
\newcommand{\be}{\begin{enumerate}}
\newcommand{\ee}{\end{enumerate}}
\newcommand{\bc}{\begin{center}}
\newcommand{\ec}{\end{center}}
\newcommand{\bd}{\begin{description}}
\newcommand{\ed}{\end{description}}
\newcommand{\I}{\item}
\newcommand{\f}{\frame}
\newcommand{\ft}{\frametitle}

\title{A Version Management System for GlueX}
\author{Mark M.\ Ito \\ \medskip Jefferson Lab}
\date{October 9, 2015}

\begin{document}

\maketitle

\begin{frame}
\ft{Outline}
  \be
\I {Introduction}
\I {The Packages}
\I {The Directory Structure}\label{section:directory}
\I {Scripts to Implement the VMS: the {\tt BUILD\_SCRIPTS} Directory}
\I {Setting Up the Shell Environment}
\I {The Build System}
\I {The Versioning System}\label{section:versioning}
\I {The Prerequisites System}\label{section:prerequisites-system}
\I {The gluex\_install System}
\ee
\end{frame}

\begin{frame}\frametitle{Introduction}
There are three fundamental areas of concern that make up all software
systems. They are:

\begin{enumerate}
\item a directory structure
\item a build system
\item a version management system
\end{enumerate}

They are all related, aspects of one affects aspects of each of the others.

\end{frame}\begin{frame}\frametitle{The Packages}

\begin{enumerate}
\item build\_scripts: scripts to manage building and the shell environment
\item Xerces-C: for reading XML files
\item CERNLIB: to support GEANT 3 simulations
\item GEANT4: simulation engine
\item ROOT: general purpose HENP toolkit
\item EVIO: CODA format data handling library
\item CCDB: Calibration Constants Database
\item JANA: event-based analysis framework
\item HDDS: detector geometry specification library 
\item sim-recon: simulation and reconstruction for GlueX
\end{enumerate}

\begin{itemize}
\item multiple versions (releases) of each of these
\item more than one version of a package may be built
\item packages depend on one or several others
\end{itemize}

\end{frame}

\begin{frame}[fragile]
  \frametitle{The Directory Structure}
{\small
\begin{verbatim}
                    $GLUEX_TOP
                    |-- build_scripts
                    |-- cernlib
                    |   `-- 2005
                    |-- hdds
                    |   |-- hdds-3.1
                    |   `-- hdds-3.2
                    |-- jana
                    |   |-- jana_0.7.2
                    |   `-- jana_0.7.3
                    |-- root
                    |   |-- root_5.34.04
                    |   `-- root_5.34.26
                    |-- sim-recon
                    |   |-- sim-recon-1.2.0
                    |   `-- sim-recon-1.3.0
                    `-- xerces-c
                        |-- xerces-c-3.1.1
                        `-- xerces-c-3.1.2
\end{verbatim}
}

\end{frame}
\begin{frame}
  \frametitle{Scripts to Implement the VMS: the {\tt BUILD\_SCRIPTS} Directory}

All scripts and makefiles to support VMS are found the in the {\tt
  build\_scripts} directory. At JLab, the full path is
$${\tt /group/halld/Software/build\_scripts}.$$
The directory can also cloned from the Git repository at GitHub. The URL is
$${\tt https://github.com/jeffersonlab/build\_scripts}.$$
Many of the scripts and makefiles require the environment variable {\tt BUILD\_SCRIPTS} to be defined.

\end{frame}
\begin{frame}\frametitle{Setting Up the Shell Environment}
\bi
\I Environment setting needed for building and using
\I Both Bourne shell and C shell supported
\ei
\end{frame}\begin{frame}\frametitle{Low-Level Environment Set-Up: {\tt gluex\_env.(c)sh}}
\bi
\I The {\tt gluex\_env.(c)sh} script will define all environment variables
\I Input: home directories of each package
\ei

\begin{center}
\begin{tabular}{|l|l|}
\hline
\bf Package & \bf Home Directory Variable \\
\hline Xerces-C & {\tt XERCESCROOT} \\
CERNLIB & {\tt CERN} \\
Geant4 & {\tt GEANT4\_HOME} \\
ROOT & {\tt ROOTSYS} \\
EVIO & {\tt EVIOROOT} \\
CCDB & {\tt CCDB\_HOME} \\
JANA & {\tt JANA\_HOME} \\
HDDS & {\tt HDDS\_HOME} \\
sim-recon & {\tt HALLD\_HOME} \\
\hline
\end{tabular}
\end{center}

\bi
\I {\tt gluex\_env.(c)sh} takes care of all the other variables.
  \bi
  \I For example, {\tt XERCES\_INCLUDE} is defined as {\tt \$XERCESCROOT/include}.
  \I Directories are added to the {\tt PATH}, {\tt LD\_LIBRARY\_PATH}, and {\tt PYTHONPATH}
  \ei
\ei
\end{frame}
\begin{frame}\frametitle{Pre-Defined Home Variables and Defaults}
\bi
\I If defined in advance, key variables will be respected.
\I If not default provided.
  \bi
  \I {\tt GLUEX\_TOP}, default =  {\tt /usr/local/gluex}
  \I {\tt BUILD\_SCRIPTS}, default = {\tt \$GLUEX\_TOP/build\_scripts}
  \I Home directories, default = {\tt \$GLUEX\_TOP/<package>/prod}
  \ei
\I See example below
\ei
\end{frame}\begin{frame}\frametitle{Consistency Checking}
\bi
\I Each home directory is checked for a prerequisites version file
\I Files list versions of prerequisite packages used at build time
\I The build-time version are checked against the environment versions
\I Warnings are printed on mismatches
\ei
\end{frame}\begin{frame}\frametitle{Cleaning the Environment: {\tt gluex\_env\_clean.(c)sh}}
\bi
\I {\tt gluex\_env.(c)sh} is sensitive to definitions hanging
around in the environment
\I Undo script: {\tt gluex\_env\_clean.(c)sh}
\ei
\end{frame}\begin{frame}\frametitle{High-Level Environment Set-Up}

\end{frame}
\begin{frame}[fragile]
  \frametitle{Example Custom Script}
\bi
\I The build of sim-recon in the user's home directory will be used.
\I All other packages be set up to use with their default builds under {\tt
    /home/gluex/gluex\_top}.
\I Note that {\tt GLUEX\_TOP} and {\tt BUILD\_SCRIPTS} are defined explicitly
\ei
\begin{verbatim}
          export GLUEX_TOP=/home/gluex/gluex_top
          export BUILD_SCRIPTS=$GLUEX_TOP/build_scripts
          export HALLD_HOME=/home/username/sim-recon
          source $BUILD_SCRIPTS/gluex_env.sh
\end{verbatim}

\end{frame}
\begin{frame}[fragile]
  \frametitle{Example Use of the Versioning System}
\bi
\I Package version information is contained in the XML file {\tt my\_versions.xml}
\I Alternate combinations of package versions can be tried by making alternate versions of the version file.
\ei
\begin{verbatim}
        export GLUEX_TOP=/home/gluex/gluex_top
        export BUILD_SCRIPTS=$GLUEX_TOP/build_scripts
        source $BUILD_SCRIPTS/gluex_env_version.sh \
            /home/username/my_versions.xml
\end{verbatim}

\end{frame}
\begin{frame}[fragile]
  \frametitle{Default Environment at JLab}
Pre-packaged scripts exist, they use the versioning system, JLab-specific

for bash:
\begin{verbatim}
source /group/halld/Software/build_scripts/gluex_env_jlab.sh
\end{verbatim}
for tcsh:
\begin{verbatim}
source /group/halld/Software/build_scripts/gluex_env_jlab.csh
\end{verbatim}

\end{frame}

\begin{frame}\frametitle{The Build System}
\bi
\I Each of the packages have their own native build system
\I Each build system has its own set of details to master.
\ei
\end{frame}\begin{frame}\frametitle{The Makefiles}
\bi
\I Makefiles invoke the native package-specific build system
\I There is a ``package makefile'' for each package ({\it e.~g.}, {\tt Makefile\_jana, Makefile\_sim-recon}).
\I Invoking make with a package makefile will build that package in the current working directory.
\I Package makefiles can be used stand-alone
\ei
\end{frame}
\begin{frame}[fragile]
  \frametitle{The Top-Level: {\tt Makefile\_all}}

\begin{verbatim}
gluex: env xerces_build cernlib_build root_build clhep_build \
       evio_build ccdb_build jana_build hdds_build \
       sim-recon_build
\end{verbatim}
\bi
\I builds all the packages necessary for using GlueX software
\I puts them in the standard directory structure
\I {{\tt Makefile\_all}} should always be invoked from the {\tt \$GLUEX\_TOP} directory
\ei
\end{frame}
\begin{frame}\frametitle{Individual Package Targets}
\bi
\I individual package targets ({\it e.~g.}, {\tt evio\_build}
and {\tt hdds\_build}) use the corresponding package makefile
\I directories are created to give the standard directory structure
\I each individual package build target can be invoked directly
\I note: CERNLIB is non-standard
\ei
\end{frame}
\begin{frame}[fragile]
  \frametitle{The Package Makefiles}\label{section:package-makefiles}
\bi
\I package makefiles are sensitive to version defining environment variables
\ei
\begin{verbatim}
      setenv JANA_VERSION 0.7.2
      setenv SIM_RECON_VERSION 1.2.0
      setenv HDDS_VERSION 3.2
\end{verbatim}
\bi
\I each makefile knows how to fetch a particular version
\I Some packages can be checked out from a source code repository
\I Instead of a version variable, a URL variable can be used
\ei
\begin{verbatim}
setenv HDDS_URL https://halldsvn.jlab.org/repos/trunk/hdds
\end{verbatim}
\bi
\I others: {\tt JANA\_URL}, {\tt SIM\_RECON\_URL}, and {\tt CCDB\_URL}
\I Special Variable for {\tt Makefile\_sim-recon}
\ei
\begin{verbatim}
        make -f $BUILD_SCRIPTS/Makefile_sim-recon \
            SIM_RECON_SCONS_OPTIONS='SHOWBUILDS=1'
\end{verbatim}
\end{frame}

\begin{frame}[fragile]
  \frametitle{Adding New Package Builds to an Existing Tree}

There are two ways:

1) Use the Top-Level {\tt Makefile\_all}

\bi
\I set up the version variables
\I invoke {\tt Makefile\_all}
\ei

\begin{verbatim}
         cd $GLUEX_TOP
         make -f $BUILD_SCRIPTS/Makefile_all gluex
\end{verbatim}

2) Use Individual Package Makefile Directly

\begin{verbatim}
cd $GLUEX_TOP/sim-recon
make -f $BUILD_SCRIPTS/Makefile_sim-recon SIM_RECON_VERSION=1.4.0
\end{verbatim}

\end{frame}

\begin{frame}[fragile]
  \frametitle{The Versioning System}\label{section:versioning}

Version File Format: an XML

\begin{verbatim}
  <gversions>
  <package name="jana" version="0.7.3"/>
  <package name="sim-recon" version="1.4.0"/>
  <package name="hdds" version="3.3"/>
  <package name="cernlib" version="2005" word_length="64-bit"/>
  <package name="xerces-c" version="3.1.1"/>
  <package name="clhep" version="2.0.4.5"/>
  <package name="root" version="5.34.26"/>
  <package name="ccdb" version="1.05"/>
  <package name="evio" version="4.3.1"/>
  </gversions>
\end{verbatim}
\end{frame}

\begin{frame}
  \frametitle{Element and Attributes}

One type of element, the {\tt package}. Attributes are:

\begin{description}
\item{\bf name}: The name of the software package.
\item{\bf version}: The version number of the package.
\item{\bf url}: A URL to be used to checkout (Subversion) or clone
  (Git) the code. The URL should point to an appropriate repository.
\item{\bf branch}: When using a Git repository, the branch to be
  checked out.
\item{\bf dirtag}: A string (directory tag) to be appended to the
  standard directory name of the package when it is built.
\item{\bf home}: Force the location of the package home directory when setting up the environment.
\end{description}

\end{frame}
\begin{frame}[fragile]
  \frametitle{Setting Up the Environment with a Version File}
{\small
\begin{verbatim}
           $BUILD_SCRIPTS/version.pl version_1.7.xml
\end{verbatim}
}
assume {\tt GLUEX\_TOP} is {\tt /home/gluex/gluex\_top}, then

{\small
\begin{verbatim}
setenv JANA_VERSION 0.7.3;
setenv JANA_HOME \
    /home/gluex/gluex_top/jana/jana_0.7.3/Linux_RHEL7-x86_64-gcc4.8.3;
setenv SIM_RECON_VERSION 1.4.0;
setenv HALLD_HOME /home/gluex/gluex_top/sim-recon/sim-recon-1.4.0;
setenv HDDS_VERSION 3.3;
setenv HDDS_HOME /home/gluex/gluex_top/hdds/hdds-3.3;
...
\end{verbatim}
}
Since you would want these commands applied to the current shell,
{\small
\begin{verbatim}
       eval `$BUILD_SCRIPTS/version.pl version_1.7.xml`
\end{verbatim}
}
\bi
\I following this step, invoke {\tt gluex\_env.(c)sh} to complete the set-up.
\I {\tt gluex\_env\_version.(c)sh} combines use of {\tt version.pl} and {\tt gluex\_env.(c)sh} (example above)
\ei
\end{frame}

\begin{frame}[fragile]
  \frametitle{Specifying Alternate Source Code Sources with a Version File}
To clone sim-recon and checkout the branch {\tt test\_stuff}, use
\begin{verbatim}
     <package name="sim-recon"
              url="https://github.com/jeffersonlab/sim-recon"
              branch="test_stuff"/>
\end{verbatim}

\end{frame}
\begin{frame}[fragile]
  \frametitle{Directory Tags}\label{section:directory-tags}

The {\tt dirtag} attribute: distinguish different builds with different prerequisite packages.
\begin{verbatim}
  <package name="hdds" version="3.3" dirtag="xerces_test"\>
\end{verbatim}
{\tt version.pl} adds a variable to the environment:
\begin{verbatim}
  setenv HDDS_DIRTAG xerces_test
\end{verbatim}
{\tt Makefile\_hdds} produces a directory named {\tt hdds-3.3\^{}xerces\_test}

\end{frame}
\begin{frame}
  \frametitle{Specifying Alternate Home Directory Locations}

A package may lie outside of the standard directory structure.
\begin{center} \tt
<package name="sim-recon" home="/home/my/sim-recon"/>
\end{center}
will cause {\tt version.pl} to generate
\begin{center} \tt
export HALLD\_HOME=/home/my/sim-recon
\end{center}
Useful for creating an environment for use; when building it you are on your own.

\end{frame}

\begin{frame}[fragile]
  \frametitle{The Prerequisites System}\label{section:prerequisites-system}
\bi
\I At build time, a version xml file is created in the home directory of
a package if that package has dependencies on others in the
system.
\I Example: \$HALLD\_HOME will have the file {\tt sim-recon\_prereqs\_version.xml}
\ei
\begin{verbatim}
                <gversion version="1.0"
                ><package name="evio" version="4.3.1"
                 /><package name="cernlib" version="2005"
                 /><package name="root" version="5.34.26"
                 /><package name="jana" version="0.7.3"
                 /><package name="hdds" version="3.3"
                 /><package name="ccdb" version="1.05"
                 /></gversion
                >
\end{verbatim}
\end{frame}

\begin{frame}
  \frametitle{How the Prerequisite File is Used}

\bi
\I At set-up time...
  \I check only if a version file with prerequisites exists
  \I each package in version file is checked for version consistency with correspoding home variable
  \I version from the home directory comes from:
\begin{enumerate}
\item {\bf Tar File}: parsed from home directory name
\item {\bf Subversion Checkout}: {\tt svn info} command is used to get the URL info
\item {\bf Git Clone}: not yet implemented!
\end{enumerate}
\ei
If a version mismatch is found, a warning is written to the screen.

\end{frame}

\begin{frame}
  \frametitle{The gluex\_install System}

\begin{itemize}
\item Described at a previous collaboration meeting
\item Uses the VMS
\item Installs all required distribution-provided packages
\end{itemize}

Tested on:

\begin{table}
$$
\begin{tabular}{|l|l|}
\hline
\bf Distribution & \bf Package Type \\
\hline
CentOS & RedHat \\
Scientific Linux & RedHat \\
Ubuntu & Debian \\
Fedora & RedHat \\
LinuxMint & Debian \\
openSUSE & RedHat \\
RedHat Enterprise & RedHat \\
\hline
\end{tabular}
$$
\end{table}

\end{frame}

\begin{frame}
  \frametitle{Summary}

\begin{itemize}
\item Automation of builds of GlueX-related software
\item Defines a standard directory structure
\item System for setting up the environment for using the software
\item Version combinations summarized succinctly in an XML file
\item XML file can be used to drive both building and environment set-up
\item Version consistency checking mechanism provided
\item Includes features for custom builds of user-specified packages
\end{itemize}
\end{frame}

\end{document}

