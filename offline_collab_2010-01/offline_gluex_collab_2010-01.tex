\documentclass[xcolor=dvipsnames]{beamer}

\usetheme{Boadilla}

\newcommand{\bi}{\begin{itemize}}
\newcommand{\ei}{\end{itemize}}
\newcommand{\be}{\begin{enumerate}}
\newcommand{\ee}{\end{enumerate}}
\newcommand{\I}{\item}
\newcommand{\f}{\frame}
\newcommand{\ft}{\frametitle}

\title{Status of the Offline Software}
\subtitle{GlueX Collaboration Meeting}
\author[M.\ Ito]{Mark M.\ Ito}
\date{January 29, 2010}
\institute[JLab]{Jefferson Lab}

\begin{document}

\f{\titlepage}

\f{\ft{Recent work} %%%%%%%%%%%%%%%%%%%%%%%%%%%%%%%%%%%%%%%%%%
\bi
\I Monte Carlo changes from Sascha S.
  \bi
  \I suppression of events with no detector hits
  \I EM shower generation in the beam collimator 
  \I Add start counter truth information to the data model
  \I Adding option to {\tt genr8\_2\_hddm} to allow a range for the $z$-vertex position
  \ei
\I Cerenkov dropped from default geometry
\I {\tt event.xml}, the HDDS geometry specification, made the authoritative file
  \bi
  \I geometry code ``made'' directly from the HDDS (code not kept in repository)
  \ei
\I FDC geometry changes from Simon {\it et~al.}
\I CDC geometry changes from Beni {\it et~al.}
\ei
}

\f{\ft{Recent work (2)} %%%%%%%%%%%%%%%%%%%%%%%%%%%%%%%%%%%%%%%%%%
\bi
\I {\tt mcsmear} should contain all random processes
  \bi
  \I policy decision
  \I Suggestion from Matt
  \I Renders reconstruction code deterministic
  \I No dependence of results for particular event on history of job
  \ei
\I problem with missing FDC and CDC hits\bi\I Richard implemented fix\ei
\I Kalman filter progress\bi\I see Simon's slides in David's talk\ei
\I 2 GB file size limit\bi\I reported by Blake, resolved by Richard\ei
\I multi-threading the DRootGeom class\bi\I David has implemented a solution\ei
\I BCAL and mcsmear changes\bi\I implemented by David\ei
\ei
}

\f{\ft{Recent work (3)} %%%%%%%%%%%%%%%%%%%%%%%%%%%%%%%%%%%%%%%%%%
\bi
\I Hall D Unix groups\bi\I simplified by Elliott\ei
\I Tracking libraries removed from HDGeant\bi\I issue raised by Richard, removed by David\ei
\I GPU effort at JLab
  \bi
  \I Jie Chen talked about use in lattice QCD farm
  \I possible applications for us (PWA: see Richard, Matt, Curtis)
  \ei
\I FDC-half cell stagger study\bi\I See Simon's talk\ei 
\I Alternate B-field studies
  \bi
  \I David has created new maps and is doing studies
  \I Ansys (from Floyd Martin) and Poisson
  \ei
\I Position dependent position smearing will be added for the FDC\bi\I functions generated by Yves, David is working on implementation\ei
\ei
}

\f{\ft{Selected tasks}
A few items from the newly-updated task list:
\bi
\I HDParSim maintenance\bi\I need a plan\ei
\I GEANT4\bi\I relatively straight-forward first-step: use HDDS geometry\ei
\I Event Viewer\bi\I need a well-defined application programmer's interface\ei
\I Formal testing of the code
  \bi
  \I standard histograms of standard reactions
  \I low-level test harnesses for software components (Dmitry R.)
  \ei
\ei
}

\f{\ft{Selected tasks (2)}
\bi
\I Calibration database\bi\I Start by taking over CLAS/PrimEx system\ei
\I Run/Beam parameters database
  \bi
  \I luminosity, beam position, magnet currents, {\it etc.}
  \I online responsibility, but need a design discussion
  \ei
\I Trigger Simulation\bi\I Underway: see Sascha S.\ei
\I Kinematic fitting\bi\I Documentation of the API\ei
\I Particle ID\bi\I detector subsystem integration\ei
\ei
}

\f{\ft{Random topics}

In random order:
\bi
\I Formal 12~GeV project schedule and milestones\bi\I ${\rm MMI} \Rightarrow$ $\overline{\rm clue}$\ei
\I Nightly build\bi\I Still not there\ei
\I Studies for detector optimization
  \bi
  \I When looking at global parameters, results may depend on subtle parts of algorithm
  \I May not be ready for some years
  \I Should we do these studies? Of course we should. Just a caveat.
  \ei
\I Multi-track events an important goal for an end-to-end solution
\I Data format: EVIO vs.\ HDDM
  \bi
  \I raw data will be in EVIO format
  \I offline uses HDDM
  \I need post-reconstruction event format (or not?)
  \ei
\ei
}

\f{\ft{Random topics (2)}

\bi
\I Geometry specification (HDDS)
  \bi
  \I Needs another round of development for ease of use: C++ API(?)
  \I Split it off: simulation, reconstruction, display should all use same API
  \I $\Rightarrow$ no need to copy around derived source code
  \ei
\I 64-bit operating systems\bi\I David has a build for CentOS\ei
\I Documentation discussion
  \bi
  \I What should we require?
  \I Form: Wiki? GlueX Notes? Software Notes?
  \ei
\I Speeding up tracking\bi\I current programs run at several Hertz, do the math\ei
\I Escaping local minima\bi\I there will be local minima\I research needed\ei
\ei
}

\f{\ft{Random topics (3)}

\bi
\I Designing/formalizing/documenting basic classes used in reconstruction\bi\I in an ideal world this would be done first\ei
\I Locking tagged releases: need a subversion upgrade
  \bi
  \I waiting on new webserver: halldweb1.jlab.org
  \ei
\I Coding conventions\bi\I Is there an appropriate model out there? CLEO, ATLAS, {\it etc.}?\ei
\I Structure for organizing MC events\bi\I we should have one\ei
\I Computing infrastructure issues
  \bi
  \I current processing crunch for CLAS at JLab
  \I How do we estimate our resource needs?
  \I How do we ramp up our resource demands to JLab Scientific Computing?
  \I What is the role of university-based computing resources?
  \ei
\ei
}

\f{\ft{Random topics (4)}

\bi
\I Charge division in the CDC\bi\I We need a study\ei
\I Operating systems: what do we want to support?
 \bi
 \I RedHat Enterprise
 \I Fedora
 \I CentOS
 \I Ubuntu
 \I Solaris
 \I Cygwin
 \I Snow Leopard
 \ei
\ei

}

\f{\ft{Conclusions}
\bi
\I Overall goal: beginning-to-end framework for reconstruction and analysis\bi\I Support high-level analysis collaboration wide in a common framework\ei
\I Organizing the effort
  \bi
  \I political units
    \bi
    \I Institutions
    \I Working Groups
    \ei
  \I What work should be done under the Offline Working Group?
  \I How to staff the tasks?
  \ei
\I Need to strengthen design and documentation
  \bi
  \I Dangers if not done
    \bi
    \I code that does not play well with others
    \I work-arounds which propagate
    \I duplication of effort
    \ei
  \I Benefits
    \bi
    \I robustness
    \I maintainability
    \I shallow learning curve for new collaborators and students
    \ei
  \ei
\ei
}

\end{document}

%%% end of latex file %%%%
