\documentclass[xcolor=dvipsnames]{beamer}

\usetheme{Boadilla}

\newcommand{\bi}{\begin{itemize}}
\newcommand{\ei}{\end{itemize}}
\newcommand{\be}{\begin{enumerate}}
\newcommand{\ee}{\end{enumerate}}
\newcommand{\I}{\item}
\newcommand{\f}{\frame}
\newcommand{\ft}{\frametitle}

\title{Status of Offline Software}
\subtitle{GlueX Collaboration Meeting}
\author[M.\ Ito]{Mark M.\ Ito}
\date{September 11, 2009}
\institute[JLab]{Jefferson Lab}

\begin{document}

\f{\titlepage}

\f{
  \ft{Outline}
  \bi
  \I Introduction
  \I Recent work (since the May Collaboration meeting)
  \I Selected tasks
  \I Random topics
  \I Conclusions
  \ei
}

\f{\ft{Introduction} %%%%%%%%%%%%%%%%%%%%%%%%%%

  \bi
  \I This talk: state/strobe of the offline
  \I Leave results of studies to appropriate working group sessions
  \I GlueX Offline Software Coordinator (GOSC): David Lawrence $\rightarrow$ MMI
    \bi
    \I Thanks to David!
    \I to Dave: Never say never to GOSC'ing again 
    \ei
  \ei
}

\f{\ft{Recent work} %%%%%%%%%%%%%%%%%%%%%%%%%%%%%%%%%%%%%%%%%%
\bi
\I Monte Carlo changes from Sascha S.
  \bi
  \I supression of events with no detector hits
  \I EM shower generation in the beam collimator 
  \I Add start counter truth information to the data model
  \I Adding option to {\tt genr8\_2\_hddm} to allow a range for the $z$-vertex position
  \ei
\I Cerenkov dropped from default geometry
\I {\tt event.xml}, the HDDS geometry specification, made the authoritative file
  \bi
  \I geometry code ``made'' directly from the HDDS (code not kept in repository)
  \ei
\I FDC geometry changes from Simon {\it et~al.}
\I CDC geometry changes from Beni {\it et~al.}
\ei
}

\f{\ft{Recent work (2)} %%%%%%%%%%%%%%%%%%%%%%%%%%%%%%%%%%%%%%%%%%
\bi
\I {\tt mcsmear} should contain all random processes
  \bi
  \I policy decision
  \I Suggestion from Matt
  \I Renders reconstruction code is deterministic
  \I No dependence of results for particular event on history of job
  \ei
\I problem with missing FDC and CDC hits\bi\I Richard implemented fix\ei
\I Kalman filter progress\bi\I see Simon's talk\ei
\I 2 GB file size limit\bi\I reported by Blake, resolved by Richard\ei
\I multi-threading the DRootGeom class\bi\I David has implemented a solution\ei
\I BCAL and mcsmear changes\bi\I implemented by David\ei
\ei
}

\f{\ft{Recent work (3)} %%%%%%%%%%%%%%%%%%%%%%%%%%%%%%%%%%%%%%%%%%
\bi
\I Hall D Unix groups\bi\I simplified by Elliott\ei
\I Tracking libraries removed from HDGeant\bi\I issue raised by Richard, removed by David\ei
\I GPU effort at JLab
  \bi
  \I Jie Chen talked about use in lattice QCD farm
  \I possible applications for us (PWA: see Richard)
  \ei
\I FDC-half cell stagger study\bi\I See Simon's talk\ei 
\I Alternate B-field studies
  \bi
  \I David has created new maps and is doing studies
  \I Ansys (from Floyd Martin) and Poisson
  \ei
\I Position dependent position smearing will be added for the FDC\bi\I functions generated by Yves, David is working on implementation\ei
\ei
}

\f{\ft{Selected tasks}
A few items from the newly-updated task list:
\bi
\I HDParSim maintenance\bi\I need a plan\ei
\I GEANT4\bi\I relatively straight-forward first-step: use HDDS geometry\ei
\I Event Viewer\bi\I need a well-defined application programmer's interface\ei
\I Formal testing of the code
  \bi
  \I standard histograms of standard reactions
  \I low-level test harnesses for software components (Dmitry R.)
  \ei
\ei
}

\f{\ft{Selected tasks (2)}
\bi
\I PWA\bi\I Getting-started guide perhaps?\ei
\I Calibration database\bi\I Start by taking over CLAS/PrimEx system\ei
\I Run/Beam parameters database
  \bi
  \I luminosity, beam position, magnet currents, {\it etc.}
  \I online responsibility, but need a design discussion
  \ei
\I Trigger Simulation\bi\I Underway: see Sascha S.\ei
\I Kinematic fitting\bi\I Documentation of the API\ei
\I Particle ID\bi\I detector subsystem integration\ei
\ei
}

\f{\ft{Random topics}

\bi
\I Formal 12~GeV project schedule and milestones\bi\I ${\rm MMI} \Rightarrow$ $\overline{\rm clue}$\ei
\I Nightly build\bi\I Still not there\ei
\I Studies for detector optimization
  \bi
  \I When looking at global parameters, results may depend on suble parts of algorithm
  \I May not be ready for some years
  \ei
\I Multi-track events an important goal
\I Data format: EVIO (raw) vs.\ HDDM
\I Geometry specification (HDDS)
  \bi
  \I Needs another round of development: ease of use: C++ API(?)
  \I split it off: simulation, reconstruction, display should all use same API
  \ei
\I 64-bit builds: David
\I Documentation discussion
  \bi
  \I eugene mentioned
  \ei
\I speeding up tracking
\I escaping local minima
\I formalizing/documenting basic classes used in reconstruction
\I locking tagged releases: subversion upgrade
  \bi
  \I waiting on new webserver: halldweb1.jlab.org
  \ei
\I design phase should be there
\I coding conventions: CLEO model?
\I structure for storing MC events
\I JLab farm issues
\I new platforms: 
 \bi
 \I centos
 \I fedora
 \I ubuntu
 \ei
\I Organizing work: offline vs. detector group
\I Action item list an active activity
\I General comment: avoid situation where ad-hoc solutions become instituionalized
  poor-design propagation
\ei

}

\f{\ft{Conclusions}
\bi
\I Lots of work done, lots to do
\I Overall goal: beginning-to-end framework for reconstruction and analysis
\I Organizing the effort
  \bi
  \I political units
    \bi
    \I Institutions
    \I Detector subsystems
    \ei
  \I design responsibility
  \I implementation manpower
  \I Dangers
    \bi
    \I ad hoc y ness
    \I duplication of effort
    \ei
  \ei
\ei
}

\end{document}

%%% end of latex file %%%%
