\documentclass[xcolor=dvipsnames]{beamer}

\usetheme{Boadilla}

\newcommand{\bi}{\begin{itemize}}
\newcommand{\ei}{\end{itemize}}
\newcommand{\be}{\begin{enumerate}}
\newcommand{\ee}{\end{enumerate}}
\newcommand{\I}{\item}
\newcommand{\f}{\frame}
\newcommand{\ft}{\frametitle}

\title{Status of the Offline Software}
\subtitle{GlueX Collaboration Meeting}
\author[M.\ Ito]{Mark M.\ Ito}
\date{May 11, 2010}
\institute[JLab]{Jefferson Lab}

\begin{document}

\f{\titlepage}

\f{
\bi
\I Dependency generation in the make scheme: revised the scheme for making dependency files in the BMS system (Mark)
\I Task priority ratings: priority ratings to the items on the Task List (Mark)
\I DANA-EVIO conversion, almost complete (Elliott)
\I PID scheme introduced (Simon)
\I Environment variable checking in the makefiles (Mark)
\I Auto-notification of build errors (Mark)
\I Noise Hits in mcsmear: default behavior of mcsmear now skip-noise-hit-generation mode (David)
\I GEANT4 for GlueX: sentiment for re-raising this issue
\I Top-level structure of our simulation and reconstruction code: ``sim-recon'' is the new name, ``src'' a child directory (Mark)

ded, a new GlueX event display, Dave Heddle

Improvements in full event reconstruction calorimeter clusters and the resulting photon objects

b1pi/omega analysis Elton

GlueX Offline Meeting, March 23, 2010

    * BMS change: binaries depend on libraries: Mark mentioned the recent change made to the binary making makefile (Makefile.bin). Now binaries should automatically be remade when libraries, upon which they depend, have changed.

    * Partial wave analysis, initial contacts: Craig has been in contact with Matt and Curtis on this.

Versioning of Calibration Constants

Mark reviewed the interim scheme (link above)

FCAL clustering, See Mihajlo's slides

Material on Splitoffs from Curtis and Matt

Documentation Structure Mark outlined some ideas

GlueX Offline Meeting, April 6, 2010

BCAL Reconstruction, David gave us a succinct outline of the reconstruction algorithm we have been using for the BCAL

DANA-to-EVIO conversion

Elliott presented his recent efforts, in collaboration with David, on a plug-in that serializes a selection of DANA objects in EVIO format.

   4. Explore action item tracking software. -> Mark 

GlueX Offline Meeting, April 20, 2010

BCAL reconstruction responsibility

New Fine-Mesh B-field Option, Simon described recent work in trying to speed up the tracking code. The code spends a lot of time in interpolation of the B-field and its derivatives. He tried an approach where the interpolation was done in advance and the results stored as a map with a much smaller grid spacing.

GlueX Offline Meeting, May 4, 2010
Minutes

   1. Mark reported that the change request, which was recently submitted to the project, for Offline Computing will have to be redone. The current version references last year's baseline BIA schedule. The request must reference the current revised schedule.

   3. David has successfully run our reconstruction on MacOS 10.6 (HDGEANT is still problematic on this platform). Changes to support this have been checked in on the trunk.

Tests of a SIMD version of DVector3

Simon described his recent re-write of the DVector3 class to take advantage of the SSE hardware present on many x86 processors. See his wiki page for details. Speed increases of a factor of two to three were seen for most member functions. The goal is to increase the speed of charged particle tracking.

    * David reported that the script to run the "b1pi" analysis is done. Mark will incorporate it into a weekly cron job. 
\ei
}

\end{document}

%%% end of latex file %%%%
