\documentclass[xcolor=dvipsnames]{beamer}

\usetheme{Boadilla}

\newcommand{\bi}{\begin{itemize}}
\newcommand{\ei}{\end{itemize}}
\newcommand{\be}{\begin{enumerate}}
\newcommand{\ee}{\end{enumerate}}
\newcommand{\I}{\item}
\newcommand{\f}{\frame}
\newcommand{\ft}{\frametitle}

\title{Status of the Offline Software}
\subtitle{GlueX Collaboration Meeting}
\author[M.\ Ito]{Mark M.\ Ito}
\date{May 11, 2010}
\institute[JLab]{Jefferson Lab}

\begin{document}

\f{\titlepage}

\f{
\bi
\I Dependency generation in the make scheme: revised the scheme for making dependency files in the BMS system (Mark)
\I Task priority ratings: priority ratings to the items on the Task List (Mark)
\I DANA-EVIO conversion, almost complete (Elliott)
\I PID scheme introduced (Simon)
\I Environment variable checking in the makefiles (Mark)
\I Auto-notification of build errors (Mark)
\I Noise Hits in mcsmear: default behavior of mcsmear now skip-noise-hit-generation mode (David)
\I GEANT4 for GlueX: sentiment for re-raising this issue
\I Top-level structure of our simulation and reconstruction code: ``sim-recon'' is the new name, ``src'' a child directory (Mark)
\I ded, a new GlueX event display (Dave Heddle)
\I Improvements in full event reconstruction calorimeter clusters and the resulting photon objects (David)
\I $b_1\pi$/$\omega$ analysis (Elton)
\ei
}

\f{
\bi

\I Make system change: binaries depend on libraries (Mark)
\I Partial wave analysis, initial contacts: Craig has been in contact with Matt and Curtis on this.
\I Interim scheme for versioning of Calibration Constants (Mark)
\I Material on Splitoffs from Curtis and Matt
\I Documentation Structure Mark outlined some ideas
\I BCAL reconstruction software responsibility: IU for now, Regina will re-acquire the job eventually
\I New Fine-Mesh B-field Option: interpolation was done in advance and the results stored as a map with a much smaller grid spacing (Simon)
\I Change request to the Project plan for ``Offline Computing'' needs to be redone; a draft has been submitted (Mark)
\I Successful build of reconstruction on MacOS 10.6 (David)
\I Tests of a SIMD version of DVector3

Simon described his recent re-write of the DVector3 class to take advantage of the SSE hardware present on many x86 processors. See his wiki page for details. Speed increases of a factor of two to three were seen for most member functions. The goal is to increase the speed of charged particle tracking.

    * David reported that the script to run the "b1pi" analysis is done. Mark will incorporate it into a weekly cron job. 
\ei
}

\end{document}

%%% end of latex file %%%%
