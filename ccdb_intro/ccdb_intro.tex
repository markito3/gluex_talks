\documentclass[xcolor=dvipsnames]{beamer}

\usetheme{Boadilla}

\newcommand{\bi}{\begin{itemize}}
\newcommand{\ei}{\end{itemize}}
\newcommand{\be}{\begin{enumerate}}
\newcommand{\ee}{\end{enumerate}}
\newcommand{\bc}{\begin{center}}
\newcommand{\ec}{\end{center}}
\newcommand{\I}{\item}
\newcommand{\f}{\frame}
\newcommand{\ft}{\frametitle}

\title{Introduction to the Calibration Database}
\subtitle{Outline of the Design}
\author[M.\ Ito]{Mark M.\ Ito}
\date{September 21, 2010}

\begin{document}

\f{\titlepage}

%============== Table Definitions  =========================================
\f{
\ft{Table Definitions}

\bi
\I Database tables express objects and relations among components in
the problem space being addressed
\I Developing, applying, and
modifying calibration constants for the various detector components.
\I There are several times associated with any set of data.
  \bi
  \I Time that the data was taken
  \I Various times at which they were calibrated
  \I Various times at which a declaration is made about which
  constants should be used for a given set of data
  \I Only this latter
meaning is relevant for this database discussion
  \ei
\ei
}

\f{
\ft{Tables and their relations}

\bc
\includegraphics[width=2.8in]{ccdb.png}
\ec

{\tiny Arrows point to table being referenced. Table at source of arrow
  contains keys identifying a row in the referenced
  table.}
}

\f{
\ft{Definition of \emph{assignment} table}
\begin{center}
\begin{tabular}{|l|l|}
\hline
\bf Column Name & \bf Column Type \\
\hline
id & int \\
time & timestamp \\
runRangeId & int \\
constantSetId & int \\
variationId & int \\
commentId & int \\
\hline
\end{tabular}
\end{center}
}

\f{
\ft{Definition of \emph{runRange} table}
\begin{center}
\begin{tabular}{|l|l|}
\hline
\bf Column Name & \bf Column Type \\
\hline
id & int \\
name & varchar \\
runMin & int \\
runMax & int \\
commentId & int \\
\hline
\end{tabular}
\end{center}
}

\f{
\ft{Definition of \emph{constSet} table}
\begin{center}
\begin{tabular}{|l|l|}
\hline
\bf Column Name & \bf Column Type \\
\hline
id & int \\
constTypeId & int \\
data & blob \\
commentId & int \\
\hline
\end{tabular}
\end{center}
}

\f{
\ft{Definition of \emph{variation} table}
\begin{center}
\begin{tabular}{|l|l|}
\hline
\bf Column Name & \bf Column Type \\
\hline
id & int \\
name & varchar \\
parentId & int \\
parentTime & datetime \\
commentId & int \\
\hline
\end{tabular}
\end{center}
}

\f{
\ft{Definition of \emph{constType} table.}
\begin{center}
\begin{tabular}{|l|l|}
\hline
\bf Column Name & \bf Column Type \\
\hline
id & int \\
name & varchar \\
rows & int \\
columns & int \\
directoryId & int \\
commentId & int \\
\hline
\end{tabular}
\end{center}
}

\f{
\ft{Definition of \emph{directory} table}
\begin{center}
\begin{tabular}{|l|l|}
\hline
\bf Column Name & \bf Column Type \\
\hline
id & int \\
name & varchar \\
parentId & int \\
commentId & int \\
\hline
\end{tabular}
\end{center}
}

\f{
\ft{Definition of \emph{columnName} table}
\begin{center}
\begin{tabular}{|l|l|}
\hline
\bf Column Name & \bf Column Type \\
\hline
constTypeId & int \\
name & varchar \\
number & int \\
type & enum \\
\hline
\end{tabular}
\end{center}
}

\f{
\ft{Logic of the tables}
\bi
\I Central table is created as a join of the assignment, runRange,
constantSet, constantType, and variation tables.
\I For each row of the assignment table one can construct:
  \bi
  \I constantType.id
  \I runRange.runMin
  \I runRange.runMax
  \I constSet.data
\ei
where constantType.id is the primary index of the constantType
table.
\I E.\ g., TOF/offset, from run 10,000 to 12,000, ``the constants''
\I Diadvantages:
  \bi
  \I Only one set of calibration constants for a given run can exist at one time.
  \I Changes made to the constants are lost.
  \I Run ranges must be non-overlapping or ambiguities about which constants should be supplied would arise.
  \ei
\ei
}

\f{
\ft{Add run range overlaps, history}
\bi
\I For the next step we add the time of constant set assignment:
\begin{itemize}
\I constantType.id
\I runRange.runMin
\I runRange.runMax
\I constSet.data
\I constantCorr.time
\end{itemize}
\I E.\ g., TOF/offset, from run 10,000 to 12,000, ``the constants'', 12/25/2015, noon
\I Now the run range data can specify overlapping ranges.
\I In case of
overlap the constants that go with the most recent time are taken.
\I Never delete old assignments $\Rightarrow$ history mechanism:
  \bi
  \I Constant set assignments that were valid at any given time can be reproduced by only accepting rows that
existed before that time.
  \I In this way the state of the table at
previous times is preserved and can be accessed.
  \ei
\ei
}

\f{
\ft{Add variations}
\bi
\I We add another column marking a variation of the assignments:
\begin{itemize}
\I constantType.id
\I runRange.runMin
\I runRange.runMax
\I constSet.data
\I constantCorr.time
\I variation.name
\end{itemize}
\I E.\ g., TOF/offset, from run 10,000 to 12,000, ``the constants'', 12/25/2015, noon, ``final calibration commissioning Fall 2015''
\I Now constants sets are assigned only for particular variations
\I Variations are in an hierarchy
\I Each variation will only deal with a relatively small number of constant set assignments.
\I If runs or
constant types are requested of a variation that are not assigned
within that variation, then the parent variation is queried,
recursively, until a valid constant set assignment is found.
\ei
}

\f{
\ft{Constant Types}
\bi
\I Constants stored as ``blobs''
\I Internal blob is a 2-D array
\I Each column has a particular type: float, int, string, etc.
\I Column labels given in columnName table
\ei
}

\f{
\ft{Directories}
\bi
\I The constType.id is the primary key for the constType table.
\I User requests come in the form a full "path" specifying the name(s) of one or more nested directories and constant type. For example:
\bi
\I CDC/vdrift
\I TOF/FADC/pedestal/method1
\I BCAL/MeVperADCCount/method1
\ei
\I Directory table contains information on the hierarchy.
\ei
}

\end{document}

% end of latex file
