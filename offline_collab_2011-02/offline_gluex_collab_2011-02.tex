\documentclass[xcolor=dvipsnames]{beamer}

\usetheme{Boadilla}

\newcommand{\bi}{\begin{itemize}}
\newcommand{\ei}{\end{itemize}}
\newcommand{\be}{\begin{enumerate}}
\newcommand{\ee}{\end{enumerate}}
\newcommand{\I}{\item}
\newcommand{\f}{\frame}
\newcommand{\ft}{\frametitle}

\title{Status of the Offline Software}
\subtitle{GlueX Collaboration Meeting}
\author[M.\ Ito]{Mark M.\ Ito}
\date{February 2, 2010}
\institute[JLab]{Jefferson Lab}

\begin{document}

\f{\titlepage}

% May 18, 2010 

\f{

\ft{Progress 1}

\bi

\I {\bf Writing out tracking error matrices.} Writing of error
matrices from track fits has been implemented. (David)

\I {\bf Calibration Database.} See Dmitry's talk.

\I {\bf Offline Computing dropped from official project management.}
Used to be part of Baseline Improvement Activities (BIA). (The
Project)

\I {\bf OSG client tools.} Now installed at JLab. The last bit was to
get the necessary ports open through firewall. (Richard)

\I {\bf EVE: an event display environment.} We had a talk on
this. Interesting idea. In principle, we can use the existing ROOT
version of the geometry. (Dmitry)

\I {\bf Correlating Hits with Tracks.} Changes made to the data model to allow identifying the particle responsible for hits associated with tracks. (Beni)

\ei
}

\f{
\ft{Progress 2}
\bi

\I {\bf Python encapsulation of C/C++ Code.} Heard a presentation on using the SWIG package to auto-generate the glue code to make the compiled code available from a Python script. Possible applications:
  \bi
  \I HDDM
  \I EVIO
  \I the new calibration database routines
  \ei

\I {\bf FLUKA simulation of neutron flux.} Simulation work on radiation damage to BCAL SiPM's. (Sascha S.)

\I {\bf Warnings largely banished from nightly builds.} (David)

\I {\bf Simulation for Compton Calorimeter}. Proposed by PrimEx folks. (David)

\I {\bf GEANT4.} Converting the Monte Carlo from GEANT3 to GEANT4. Is
the time now? (David)

\I {\bf Time and energy smearing for the start counter added in
  mcsmear.} (Simon)

\ei
}
\f{
\ft{Progress 3}
\bi

\I {\bf Tagger hall geometry/simulation.} Work to characterize neutron
fluence for the microscope SiPM's. Updated tagger hall geometry
available. (Sascha S.)

\I {\bf Random number seed policy} Will use {\tt /dev/urandom} as a basis
for generating seeds. (Richard)

\I {\bf Parallelism Workshop}



David gave some impressions from last week's [38]Workshop on
Parallelism in Experimental Nuclear Physics. Fifty five people
registered and a like number attended. There were lots of interesting
and informative talks, featured among them talks from folks outside the
JLab community. See [39]David's email for his summary for the
participants and a link to the virtual machine that has the hands-on
examples loaded on it.

Review of IT Readiness for the 12 GeV Era

* need an outline of remarks by all speakers by end of month
* the review will be around the 2nd week of March
* Graham will send around a spreadsheet, asking for updates to
computing requirements from various groups
* scheduled speakers: Bob Michaels from Hall A, Dennis Weygand from
Hall B, Mark Ito from Hall D, a representative from Hall C--TBA,
and Graham with an overview

Changes to TOF Code

Beni described recent changes he has made to the HDDM data model to
accommodate information from the Monte Carlo about "truth" information
related to time-of-flight hits. In some cases, new elements were
introduced to be able to keep both generated and detected information,
and some elements were expanded to store information about the particle
producing the hits.

A lot of discussion followed on the general idea of keeping generated
quantities ("truth") and detected quantities ("hits") separate in our
analysis. Some of Beni's changes seemed to violate this separation.
Beni, Mark, and David agreed to meet after the meeting to discuss
alternate approaches that preserve the information desired, but make
the above mentioned separation more explicit.

Tracking Issues

Kei presented a list of bugs/anomalies that he has come across in
running our reconstruction code. See [44]his talk for details. He
mentioned four items:

0. event processing hangs every ~1000 events for his event topology
1. pi+ events reconstructed as proton show “spike” in momentum at ~0.5
GeV/c
2. some events have direction of momentum “reversed”
3. values of variables are set to non-physical values when they cannot
be defined for a given track (leads to side-effects if not checked for
explicity)

Examination of the reconstruction algorithm as inherited from KLOE

    Mark mentioned that he had seen this phrase in the minutes of one of
    the calorimeter meetings (attributed to Zisis) and he wondered if this
    was an issue the offline group should be tracking or discussing or
    hearing about. Ryan commented that Dan Bennett and Matt Shepherd had
    made improvements to the code, but there are still some mysteries in
    the code that someone in the collaboration should figure out. Elton
    mentioned that Regina was starting to look at maintaining the BCAL
    code. David will check with Regina folks on the status of the effort.

\ei
}

\f{
\ft{Workshop Talks}
\small
\begin{tabular}{| p{2.8 in} | l |}
\hline
   Compilers and Performance Tools & Jeff Arnold (Intel) \\
   JLab Scientific Computing & Sandy Philpott (JLab) \\
   Introduction to GPUs & Balint Joo (JLab) \\
   GPU-based tracking for ATLAS L2 Trigger & Dmitry Emeliyanov (RAL) \\
   Amplitude Analysis on GPUs & Matthew Shepherd (IU) \\
   Parallel Processing in Particle and High Energy Physics (Distributed Computing with CLARA) & Dennis Weygand (JLab) \\
   Adding Multi-core Support to CMS' Legacy Data Processing System & Chris Jones (FNAL) \\
   The (multi-threaded) JANA Reconstruction Framework & David Lawrence (JLab) \\
   Parallelism in GEANT4 & Makoto Asai (SLAC) \\
   Parallelism in DAQ Systems & David Abbott (JLab) \\
   Vectorization with SIMD & Simon Taylor (JLab) \\
\hline
\end{tabular}
}

\f{
\ft{Tutorial Talks}
\small
\begin{tabular}{| p{2.8 in} | l |}
\hline
   Overview of GlueX Data Rates & Sascha Somov (JLab) \\
   Overview of GlueX Offline Computing & Mark Ito (JLab) \\
   GlueX Simulation & Richard Jones (UConn) \\
   Amplitude Analysis Tutorial & Matthew Shepherd (IU) \\
   DANA: The Hall-D implementation of JANA & David Lawrence (JLab) \\
   GlueX Reconstruction & Simon Taylor (JLab) \\
   Calibration Database & Dmitry Romanov (MEPHI) \\
   The ded Event Viewer & Andrew Blackburn (CNU) \\
\hline
\end{tabular}
}


\f{
\ft{Progress 4}
\bi

\I {\bf Documentation Policy} Written down, captures current practices, introduced FAQ\cite{doc-policy} (Mark)

% August 10, 2010

\I {\bf b1pi analysis in cron job}. Plots that are now produced automatically, weekly basis.\cite{b1pi-auto} (David, Mark)

\I {\bf Hall-D File Formats} Lots of discussion recently.\cite{halld-file-format} (all)

% August 24, 2010

\I {\bf HDGEANT auto-smearing}. Run mcsmear automatically, new feature.\cite{auto-smear} (David)

\ei

}

\f{
\ft{Summary and Remarks\footnote{copied from last time}}
\bi
\I Progress in many areas
\I Major current challenges
\be
\I Extraneous clusters from hadronic interactions
\I Speed of charged particle tracking
\bi
\I Technological fix?
\I Algorithmic fix?
\ei
\ee
\I Minor current challenges
\be
\I Output format? HDDM? EVIO? ROOT? All of the above?
\I Global timing and PID
\I Doing a native build on 64-bit OS's (e.\ g., Snow Leopard, Linux x86\_64)
\ee
\I Long-term challenges: see the Prioritized Task List
\I Need for better documentation
\ei
}

\setbeamertemplate{bibliography item}[text]

\f{
\ft{References}
\begin{thebibliography}{99}
\tiny
\bibitem{save-tracking}$[[$HOWTO save tracking results to an HDDM or EVIO file for later playback$]]$
\bibitem{Simon-on-tracking} http://argus.phys.uregina.ca/cgi-bin/private/DocDB/ShowDocument?docid=1576
\bibitem{kinematic}$[[$HOWTO do a kinematic fit for etapi0p events$]]$
\bibitem{pre-built}$[[$HOWTO use a pre-built release$]]$
\bibitem{uniform-field} http://argus.phys.uregina.ca/cgi-bin/private/DocDB/ShowDocument?docid=1556
\bibitem{sim-recon-wiki}$[[$Sim-Recon Tagged Releases$]]$
\bibitem{time-zero}$[[$How HDGeant defines time-zero for physics events$]]$
\bibitem{multi-core}$[[$Media:20100615\_48cores.pdf$]]$
\bibitem{new-jana}https://mailman.jlab.org/pipermail/halld-offline/2010-June/000314.html
\bibitem{ccdb} http://argus.phys.uregina.ca/cgi-bin/private/DocDB/ShowDocument?docid=1541
\bibitem{grid-tools}http://markito3.wordpress.com/2010/06/23/notes-on-grid-client-tools-meeting/, http://zeus.phys.uconn.edu/UConn-OSG/client-services.html
\bibitem{bcal-smear}http://argus.phys.uregina.ca/gluex/DocDB/0015/001552/001/BCAL\_Software\_progress\_062910.pdf
\bibitem{mantis}https://halldnew.jlab.org/mantisbt/
\bibitem{jana-plugin}$[[$Media:20100713\_plugins.pdf$]]$
\bibitem{ded-doc}$[[$HOWTO ded: Install \& Run$]]$
\bibitem{perl-particles}https://mailman.jlab.org/pipermail/halld-offline/2010-July/000330.html
\bibitem{doc-policy}$[[$Software Documentation Structure$]]$ and $[[$GlueX Offline FAQ$]]$
\bibitem{b1pi-auto}$[[$Weekly Tests of GlueX Software$]]$
\bibitem{halld-file-format}$[[$GlueX Offline Meeting, August 10, 2010$]]$ and $[[$GlueX Offline Meeting, August 24, 2010$]]$
\bibitem{auto-smear}https://mailman.jlab.org/pipermail/halld-offline/2010-August/000348.html
\end{thebibliography}
}

\end{document}

%%% end of latex file %%%%
