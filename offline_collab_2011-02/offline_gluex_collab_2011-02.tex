\documentclass[xcolor=dvipsnames]{beamer}

\usetheme{Boadilla}

\newcommand{\bi}{\begin{itemize}}
\newcommand{\ei}{\end{itemize}}
\newcommand{\be}{\begin{enumerate}}
\newcommand{\ee}{\end{enumerate}}
\newcommand{\I}{\item}
\newcommand{\f}{\frame}
\newcommand{\ft}{\frametitle}

\title{Status of the Offline Software}
\subtitle{GlueX Collaboration Meeting}
\author[M.\ Ito]{Mark M.\ Ito}
\date{February 2, 2010}
\institute[JLab]{Jefferson Lab}

\begin{document}

\f{\titlepage}

% May 18, 2010 

\f{

\ft{Progress 1}

\bi

\I {\bf Writing out tracking error matrices.} Writing of error
matrices from track fits has been implemented. (David)

\I {\bf Calibration Database.} See Dmitry's talk.

\I {\bf Offline Computing dropped from official project management.}
Used to be part of Baseline Improvement Activities (BIA). (The
Project)

\I {\bf OSG client tools.} Now installed at JLab. The last bit was to
get the necessary ports open through firewall. (Richard)

\I {\bf EVE: an event display environment.} We had a talk on
this. Interesting idea. In principle, we can use the existing ROOT
version of the geometry. (Dmitry)

\I {\bf Correlating Hits with Tracks.} Changes made to the data model to allow identifying the particle responsible for hits associated with tracks. (Beni)

\ei
}

\f{
\ft{Progress 2}
\bi

\I {\bf Python encapsulation of C/C++ Code.} Heard a presentation on using the SWIG package to auto-generate the glue code to make the compiled code available from a Python script. Possible applications:
  \bi
  \I HDDM
  \I EVIO
  \I the new calibration database routines
  \ei

\I {\bf FLUKA simulation of neutron flux.} Simulation work on radiation damage to BCAL SiPM's. (Sascha S.)

\I {\bf Warnings largely banished from nightly builds.} (David)

\I {\bf Simulation for Compton Calorimeter}. Proposed by PrimEx folks. (David)

\I {\bf GEANT4.} Converting the Monte Carlo from GEANT3 to GEANT4. Is
the time now? (David)

\I {\bf Time and energy smearing for the start counter added in
  mcsmear.} (Simon)

\ei
}
\frame<1>[label=ProgressThree]{
\ft{Progress 3}
\bi

\I {\bf Tagger hall geometry/simulation.} Work to characterize neutron
fluence for the microscope SiPM's. Updated tagger hall geometry
available. (Sascha S.)

\I {\bf Random number seed policy} Will use {\tt /dev/urandom} as a basis
for generating seeds. (Richard)

\I {\bf Parallelism Workshop} (David and Ed Brash)
  \bi
  \I Fifty five people attended
  \I Full program
  \I Virtual machine used for tutorial still available (free)
  \ei

\I {\bf Review of IT Readiness for the 12 GeV Era}
  \bi
  \I Internal JLab review, 2nd week of March
  \I Graham Heyes has solicited input
  \I Presentations from the Halls (MMI for Hall D)
  \I Informal discussion?
  \ei

\I {\bf Changes to TOF Code to add MC-only information} Changes to HDDM model and hit generation in HDGEANT. (Beni)

\I {\bf Tracking Issues} Discussion of bugs/anomalies in the tracking code (Kei)

\ei
}

\f{
\ft{Workshop Talks}
\small
\begin{tabular}{| p{2.8 in} | l |}
\hline
   Compilers and Performance Tools & Jeff Arnold (Intel) \\
   JLab Scientific Computing & Sandy Philpott (JLab) \\
   Introduction to GPUs & Balint Joo (JLab) \\
   GPU-based tracking for ATLAS L2 Trigger & Dmitry Emeliyanov (RAL) \\
   Amplitude Analysis on GPUs & Matthew Shepherd (IU) \\
   Parallel Processing in Particle and High Energy Physics (Distributed Computing with CLARA) & Dennis Weygand (JLab) \\
   Adding Multi-core Support to CMS' Legacy Data Processing System & Chris Jones (FNAL) \\
   The (multi-threaded) JANA Reconstruction Framework & David Lawrence (JLab) \\
   Parallelism in GEANT4 & Makoto Asai (SLAC) \\
   Parallelism in DAQ Systems & David Abbott (JLab) \\
   Vectorization with SIMD & Simon Taylor (JLab) \\
\hline
\end{tabular}
}

\f{
\ft{Tutorial Talks}
\small
\begin{tabular}{| p{2.8 in} | l |}
\hline
   Overview of GlueX Data Rates & Sascha Somov (JLab) \\
   Overview of GlueX Offline Computing & Mark Ito (JLab) \\
   GlueX Simulation & Richard Jones (UConn) \\
   Amplitude Analysis Tutorial & Matthew Shepherd (IU) \\
   DANA: The Hall-D implementation of JANA & David Lawrence (JLab) \\
   GlueX Reconstruction & Simon Taylor (JLab) \\
   Calibration Database & Dmitry Romanov (MEPHI) \\
   The ded Event Viewer & Andrew Blackburn (CNU) \\
\hline
\end{tabular}
}


\againframe<2>{ProgressThree}

\f{
\ft{Summary and Remarks}
\bi
\I Progress in many areas
\I Current issues
  \be
  \I Extraneous clusters from hadronic interactions
  \I Speed of charged particle tracking
  \I General robustitude of reconstruction
  \I Revision of computing plan
  \ee
\I Feedback on problems important!
\ei
}

\end{document}

%%% end of latex file %%%%
