\documentclass[xcolor=dvipsnames]{beamer}

\usetheme{Boadilla}

\newcommand{\bi}{\begin{itemize}}
\newcommand{\ei}{\end{itemize}}
\newcommand{\be}{\begin{enumerate}}
\newcommand{\ee}{\end{enumerate}}
\newcommand{\bc}{\begin{center}}
\newcommand{\ec}{\end{center}}
\newcommand{\I}{\item}
\newcommand{\f}{\frame}
\newcommand{\ft}{\frametitle}

\title{Overview of GlueX Offline Computing}
\subtitle{at the Parallelism in Experimental Nuclear Physics Workshop}
\author[M.\ Ito]{Mark M.\ Ito}
\date{January 7, 2011}
\institute[JLab]{Jefferson Lab}

\begin{document}

\f{\titlepage}

\f{
\ft{Geometry}
  \bi
  \I implemented in XML: HDDS
  \ei
}

\f{
\ft{Simulation}
\bi
\I GEANT3-based: HDGEANT
  \bi
  \I Geometry information auto-coded into fortran code from HDDS information
  \I Hits coded separately
  \I Output in HDDM format
  \ei
\I Experimental resolution added in separate stage: hdgeant
  \bi
  \I HDDM in, HDDM out
  \ei
\ei
}

\f{
\ft{Reconstruction}
\bi
\I JANA
  \bi
  \I multi-threaded: each thread a separate event stream
  \I algorithms for different detectors implemented as "factories"
  \ei
\I ROOT used for some general utilities
\I Hooks for user code
  \bi
  \I user's class inherits from abstract base class
  \I must be registered with the framework
  \I multiple user classes possible
  \ei
\I Plug-in mechanism
  \bi
  \I e. g., define user class at run time
  \ei
\ei
}

\f{
\ft{Partial Wave Analysis}
\bi
\I AmpTools
\I Ruby-PWA
\ei
}

\f{
\ft{Calibration Database}
\bi
\I relational database
\I based on CLAS experience (Hall B, JLab)
\I improvements over that design
\ei
}

\f{
\ft{Data Format}
\bi
\I Raw data: EVIO
\I Simulation output: HDDM
\I Reconstruction output:
  \bi
  \I HDDM
  \I EVIO
  \I ROOT trees
  \ei
\ei
}

\f{
\ft{Utilities}
\bi
\I XML parsing: Xerces
\I Source code management: subversion
\I Source code documentation: doxygen
\I Building scripts: GNU Make
\I Database: MySQL
\I General documentation
  \bi
  \I GlueX Notes: DocDB
  \I Webpages: mediawiki
  \ei
\ei
}

\f{
\ft{Computing}
\bi
\I Simulation
  \bi
  \I Grid
  \ei
\I Reconstruction
  \bi
  \I JLab Farm
  \ei
\I PWA
  \bi
  \I Grid
  \I GPU farms
  \ei
\ei
}

\end{document}

% end of latex file
